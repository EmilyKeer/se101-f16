\documentclass{article}
\usepackage[top=1in, bottom=1in, left=1.5in, right=1.5in]{geometry}
\usepackage{graphicx}
\usepackage{hhline}
\usepackage{multirow}
\usepackage{url}

\begin{document}
\title{SE101---Project}
\author{Rollen D'Souza, Patrick Lam, Mahesh Tripunitara}
\date{Fall 2016\\Last Updated \today}
\maketitle

\section*{Introduction}
In this lab you are given an opportunity to follow the full software development cycle and to be creative with the microcontroller board.  This lab is flexible. However, your mark is proportional to both the difficulty of the project you attempt, as well as the performance of your execution.  Please use both the microcontroller board as well as the Orbit Booster Pack.  Keep in mind that you should be able to complete the project with minimal assistance.

\section*{Requirements}
\begin{itemize}
	\item{In groups of 2 students, submit a proposal of the project you would like to complete. The document must:}
		\begin{itemize}
			\item Be a PDF.
			\item Clearly state full name, ID number, and Quest/WATIAM ID for both students in the group.
			\item Be at most 1 page long. 
		\end{itemize}
	\item{The proposal document must describe the following:}
		\begin{itemize}
			\item Your project. What will it do?
			\item The major software components of your project (e.g. implement LCD user interface, implement accelerometer data collection).
			\item The hardware components you intend to use in the project.
			\item The challenges you anticipate.  For example: if you are using the LCD screen, you may anticipate the challenge of drawing and moving a complex object.
		\end{itemize}
	\item{You must \emph{develop} and submit the project using \texttt{svn}.  Marks will be deducted for not using source control effectively.  That is, we expect to see at least two non-trivial commits.  Trivial commits are those that consist of only changes to whitespace.}

\end{itemize}

\subsection*{Important Dates}
\begin{tabular}{rl}
Proposal Due Date & October 24, 2016 @ 23:59\\
Proposal Feedback & October 27, 2016 @ 11:30\\
Project Demo Due Date & December 5, 2016 @ 11:30
\end{tabular}

\section*{Marking Scheme}
\begin{center}
\begin{tabular}[c]{cr|c}
& \textbf{Criteria} & \textbf{Mark Contribution} \\ \hline

& Project Proposal & 10 \\ \hline

\multirow{3}{*}{Execution}
	& Low Difficulty & 10 \\
	& Average Difficulty & 10 \\
	& Above Average Difficulty & 10 \\ \hline

\multirow{5}{*}{Style} 
	& Naming & 2 \\
    & Whitespace Usage & 2 \\
	& Easy to Modify & 2 \\
	& Use of Control Flow and Types & 2 \\
	& Use of Source Control & 2 \\ \hhline{==|=}
	
& Total & 50 \\ \hhline{==|=}
& Bonus & 5
\end{tabular}
\end{center}

\section*{Submission}
You must submit your assignment using the ECE SVN (subversion) repository.  You are graded on your use of Subversion.  We expect to see at least two non-trivial commits. Trivial commits are those that consist of only changes to whitespace.

Please see Piazza for more information on using SVN.

Please commit your solution to:

\begin{center}
\url{https://ecesvn.uwaterloo.ca/courses/se101/2016/students/YOUR\_WATID/project}
\end{center}

where you replace ``YOUR\_WATID'' with your WatID (e.g. p23lam). Email Patrick Lam if you can't commit to that address.

\section*{Tips}
\begin{itemize}
	\item The moment you have a working program, commit it! If things go wrong, you can always go back.
	\item Expect to stumble upon unexpected hurdles.  Always give yourself an ample amount of time to finish the project!	
\end{itemize}

\begin{flushleft}
\begin{thebibliography}{1}

\end{thebibliography}
\end{flushleft}

\end{document}
